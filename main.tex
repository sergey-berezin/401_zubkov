\documentclass{article}

\usepackage[english, russian]{babel}
\usepackage[letterpaper,top=2cm,bottom=2cm,left=3cm,right=3cm,marginparwidth=1.75cm]{geometry}

\usepackage{amsmath}
\usepackage{graphicx}
\usepackage{amsfonts}
\usepackage{amssymb}
\usepackage{amsthm}
\usepackage{mathtools}
\usepackage[colorlinks=true, allcolors=blue]{hyperref}

\newcommand{\bfv}[1]{\mathbf{#1}}
\newcommand{\dd}[1]{\dot{#1}}
\newcommand{\dvp}[3]{#1\,\times\,[\,#2\,\times\,#3\,]}
\newcommand{\dv}[1]{\nabla v(#1)}
\newcommand{\ddv}[1]{\mathrm{D}[v](#1)}
\newcommand{\dr}{\delta \bfv{r}}
\newcommand{\dn}{\delta \bfv{n}}
\newcommand{\om}[1]{\mathrm{o}(#1)}
\newcommand{\dprod}[2]{\langle #1, #2 \rangle}
\newcommand{\T}[1]{#1^\mathrm{T}}
\newcommand{\matr}[1]{\mathrm{#1}}
%\title{Your Paper}
%\author{You}

\begin{document}

\section{Постановка задачи}
Рассмотрим исходную систему бихарактеристик луча в переменных $\bfv{r}$ и $\bfv{n}$:
\begin{equation} \label{eq1}
\begin{cases}
\dd{\bfv{r}} = v(\bfv{r})\,\bfv{n}\\
\dd{\bfv{n}} = \dvp{\bfv{n}}{\bfv{n}}{\dv{\bfv{r}}}\\
\end{cases}
\end{equation}
с начальными условиями $\bfv{r}|_{\tau=0} = \bfv{r}_0$ и $\bfv{n}|_{\tau=0} = \bfv{n}_0$\\\\
Рассмотрим малое возмущение для $\bfv{r}$ и $\bfv{n}$:
\begin{align*}
\bfv{r_1} &= \bfv{r} + \dr   &   \bfv{n_1} &= \bfv{n} + \dr
\end{align*}
Будем считать, что в линейном приближении система \eqref{eq1} верна для переменных $(\bfv{r_1}\,\bfv{n_1})$, то есть: 
\begin{equation} \label{eq2}
\begin{cases}
\dd{\bfv{r_1}} = v(\bfv{r_1})\,\bfv{n_1} + \om{\|\dr\| + \|\dn\|}\\
\dd{\bfv{n_1}} = \dvp{\bfv{n_1}}{\bfv{n_1}}{\dv{\bfv{r_1}}} + \om{\|\dr\| + \|\dn\|}\\
\end{cases}
\end{equation}\\
Разложим в ряд Тейлора функции $v(\bfv{r})$ и $\dv{\bfv{r}}$:
\begin{gather*} 
v(\bfv{r_1}) = v(\bfv{r}) + \dprod{\dv{\bfv{r}}}{\dr} + \om{\|\dr\|}\\ 
\dv{\bfv{r_1}} = \dv{\bfv{r}} + \ddv{\bfv{r}}\,\dr + \om{\|\dr\|}
\end{gather*}
Подставим полученные разложения в систему \eqref{eq2}, воспользовавшись формулой Лагранжа для двойного векторного произведения: 
\begin{equation} \label{eq3}
\begin{cases}
\dd{\bfv{r}} + \dd{\dr} = v(\bfv{r})\,\bfv{n} + \dprod{\dv{\bfv{r}}}{\dr}\,\bfv{n} + v(\bfv{r})\,\dn + \om{\|\dr\| + \|\dn\|}\\
\dd{\bfv{n}} + \dd{\dn} = \dvp{\bfv{n}}{\bfv{n}}{\dv{\bfv{r}}} + \dvp{\bfv{n}}{\bfv{n}}{\ddv{\bfv{r}}} \ +\\
\qquad+ \dprod{\dv{\bfv{r}}}{\dn}\,\bfv{n} - 2\,\dprod{\bfv{n}}{\dn}\,\dv{\bfv{r}} + \dprod{\bfv{n}}{\dv{\bfv{r}}}\,\dn + \om{\|\dr\| + \|\dn\|}\\
\end{cases}
\end{equation}\\
Используя равенства системы \eqref{eq1}, перепишем \eqref{eq3} в следующем виде:
\begin{equation*}
\begin{cases}
\dd{\dr} = \dprod{\dv{\bfv{r}}}{\dr}\,\bfv{n} + v(\bfv{r})\,\dn + \om{\|\dr\| + \|\dn\|}\\
\dd{\dn} = \dvp{\bfv{n}}{\bfv{n}}{\ddv{\bfv{r}}\,\dr} + \dprod{\dv{\bfv{r}}}{\dn}\,\bfv{n} - 2\,\dprod{\bfv{n}}{\dn}\,\dv{\bfv{r}} + \dprod{\bfv{n}}{\dv{\bfv{r}}}\,\dn + \om{\|\dr\| + \|\dn\|}
\end{cases}
\end{equation*}
Что эквивалентно следующей записи:
\begin{equation} \label{eq4}
\begin{cases}
\dd{\dr} = \T{\bfv{n}}\dv{\bfv{r}}\,\dr + v(\bfv{r})\,\dn + \om{\|\dr\| + \|\dn\|}\\
\dd{\dn} = \T{\bfv{n}}\bfv{n}\,\ddv{\bfv{r}}\,\dr - \dprod{\bfv{n}}{\bfv{n}}\ddv{\bfv{r}}\,\dr + \T{\bfv{n}}\dv{\bfv{r}}\,\dn \ - \\ \qquad -\ 2\,\T{\dv{\bfv{r}}}\bfv{n}\,\dn + \dprod{\bfv{n}}{\dv{\bfv{r}}}\,\dn + \om{\|\dr\| + \|\dn\|}
\end{cases}
\end{equation}\\
В силу сферичности фронта положим $\dr = \matr{P}(\tau)\,\dn_0$ и $\dn = \matr{Q}(\tau)\,\dn_0$ и оставим только линейные $\delta$-члены. В результате система \eqref{eq4} примет вид:
\begin{equation} \label{eq5}
\begin{cases}
\dd{\matr{P}} = \T{\bfv{n}}\dv{\bfv{r}}\,\matr{P} + v(\bfv{r})\,\matr{Q}\\
\dd{\matr{Q}} = (\T{\bfv{n}}\bfv{n}\,\ddv{\bfv{r}} - \dprod{\bfv{n}}{\bfv{n}}\,\ddv{\bfv{r}})\,\matr{P} + (\T{\bfv{n}}\dv{\bfv{r}} - 2\,\T{\dv{\bfv{r}}}\bfv{n} + \dprod{\bfv{n}}{\dv{\bfv{r}}}\,\matr{I})\,\matr{Q} 
\end{cases}
\end{equation}
Начальные условия: $\matr{P}(0) = \matr{\O}$ и $\matr{Q}(0) = \matr{I}$
\end{document}